\section{Teoria sobre BCH}

Códigos cíclicos BCH primitivos são completamente caracterizados por dois parâmetros: $m$ e $t$, ambos naturais, de modo que se usará a notação $BCH(m,t)$ para os individualizar. $m$ é tal que $n=2^m-1$ é o tamanho de bloco do código e $t$ é a distância mínima planejada (necessariamente menor ou igual à distância mínima real). 

O polinômio gerador de $BCH(m,t)$ é obtido pelo mínimo múltiplo comum entre os polinômios mínimos de $\alpha^1, ..., \alpha^{2t}$, em que $\alpha$ é um elemento primitivo de $GF(2^m)$.

A decodificação desses códigos, diferentemente de outros cíclicos e não-cíclicos, não exige que se memorizem associações síndrome-erro e é computacionalmente eficiente, apresentando complexidade $O(n)$ demonstrada em \ref{complexidade_decod}.

O algoritmo utilizado foi retirado de \cite{ref:algoritmo-berlekamp} e é conhecido como "Berlekamp-Massey".