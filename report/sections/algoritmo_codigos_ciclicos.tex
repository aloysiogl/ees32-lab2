\section{Algoritmo códigos cíclicos}
A algoritmo desenvolvido pode ser dividido nas seguintes etapas: obtenção dos polinômios geradores, codificação, e decodificação. Coda uma dessas etapas esta descrita nas subseções abaixo.

\subsection{Obtenção dos polinômios geradores}
Foram escolhidos, inicialmente, cinco tipos de códigos deferentes com duplas $(k, n)$ mostradas a seguir contidas em $S = \{(6, 10), (7, 12), (8, 14), (9, 15), (9, 16)\}$. Para cada elemento de $S$ foram gerados todos os polinômios geradores $g_{ij}$ em que $i$ é o índice em $S$ e $j$ o índice de um polinômio qualquer quer gere um código $S_i$. Com este fim, foi utilizada a função \textit{cyclpoly} do MATLAB.

Para cada $i$ o conjunto $g_{ij}$ guarda os polinômios
