\section{Algoritmo códigos cíclicos}
A algoritmo desenvolvido pode ser dividido nas seguintes etapas: obtenção dos polinômios geradores, codificação, e decodificação. Coda uma dessas etapas esta descrita nas subseções abaixo.

\subsection{Obtenção dos polinômios geradores}
Foram escolhidos, inicialmente, cinco tipos de códigos deferentes com duplas $(k, n)$ mostradas a seguir contidas em $S = \{(6, 10), (7, 12), (8, 14), (9, 15), (9, 16)\}$. Para cada elemento de $S$ foram gerados todos os polinômios geradores $g_{ij}$ em que $i$ é o índice em $S$ e $j$ o índice de um polinômio qualquer quer gere um código $S_i$. Com este fim, foi utilizada a função \textit{cyclpoly} do MATLAB.

Para cada $i$ dentre todos os $j$ escolhe-se o polinômio $g_{ij}$ que gera o código de maior distância mínima. Para isso, geraram-se todas as palavras informação e calculou-se o peso de cada palavra código a fim de achar a distância mínima. A complexidade desta etapa é $\mathcal{O}(2^kn^{3.37})$. Considerou-se o pior caso da multiplicação de matriz $\mathcal{O}(n^{2.37})$, uma vez que para o cálculo da palavra código pela palavra informação utilizou-se a multiplicação matricial. O algoritmo descrito foi implementado na linguagem \textit{Python}.

A principal dificuldade desta parte do código foi achar a distância mínima, no entanto, muito se pôde aproveitar da atividade anterior. Dessa forma a dificuldade desta etapa não foi tão grande.

\subsection{Codificação}

A codificação consistiu simplesmente de uma multiplicação de polinômios, a qual foi feita em \textit{Python} utilizando a função \textit{polymul} da biblioteca \textit{numpy}. Dessa forma a complexidade de codificação do sistema ficou $\mathcal{O}(n^2)$ para uma palavra código de tamanho $n$.

Vale ressaltar que o canal utilizado foi exatamente o mesmo do experimento anterior, por isso, sua implementação nõa será detalhada neste documento.

\subsection{Decodificação}

Esta foi a parte mais desafiadora do experimento, uma vez que esta parte contém a envolve a correção de erros. A decodificação consiste em um divisão polinomial da palavra código pelo polinômio gerador, no entanto se houver resto (síndrome), deve haver correção de erros.

Vale ressaltar que o código foi implementado em \textit{Python} e que a divisão de polinômios foi implementada seguindo o algoritmo de divisão manual com soma resto 2. A complexidade final do algoritmo de divisão polinomial ficou $\mathcal{O}(n^2)$.

Para corrigir os erros foi implementado um algoritmo realiza rotações para ser capaz de utilizar síndromes associadas apenas a erros na primeira posição. O pseudocódigo para tal algoritmo é mostrado em Algoritimo \ref{alg:decode}.

\begin{algorithm}
	\caption{Decodificação}\label{alg:decode}
	\begin{algorithmic}[Message]
		\Procedure{$decode$}{$message$}
			\State $sind \gets calc\_sindrome(message)$
			
			\While {$sind \neq all\_zeros$}
				\If {$sind \in end\_one\_sinds$}
					\State $change\_first(message)$
					\State $sind \gets calc\_sindrome(message)$
				\Else
				\State $rotate(message, -1)$
				\State $rotate(sind, -1)$
					\If{$sind.first = 1$} 
						\State $sind.first \gets 0$
						\State $sind \gets sind+g$
					\EndIf
				\EndIf
			\EndWhile
		\State $message \gets unrotate(message)$\\
		\Return $divide(message, g)$
		\EndProcedure
	\end{algorithmic}
\end{algorithm}

A parte mais desafiadora da atividade foi encontrar o conjunto de síndromes, representado no algoritmo por $end\_one\_sinds$. Consta em Algoritmo \ref{alg:update_sindromes} o código utilizado para encontrar um conjunto de síndromes associadas a erros que começam em 1 de maneira a garantir que o código de Algoritimo \ref{alg:decode} termine em ao menos $n$ ciclos do laço mais externo.

\begin{algorithm}
	\caption{Decodificação}\label{alg:update_sindromes}
	\begin{algorithmic}[Message]
		\Procedure{$update\_sindrom	es$}{$ $}
		\State $sind\_map \gets \emptyset$
		\For {$i \in \{1, len(codeword)\}$}
			\State $errs \gets list(errors, weigth=i)$
			\For {$err \in errs$}
				\State $err_key \gets calc\_sindrome(err)$
				\If {$err\_key=all\_zeros$}
					\State $continue$
				\EndIf
				
				\If {$err\_key \notin sind\_map $ $\cup$\\ $weigth(sind\_map[err\_key]) = i$ $\cap$ \\ $error\_ends\_with\_zero(sind\_map[err\_key])$}
				\State $sind\_map[err\_key] \gets (err.first, i)$
				\EndIf
			\EndFor
		\EndFor
		\For {$key \in sind\_map$}
			\If{$error\_ends\_with\_one(sind\_map[err\_key])$}
				\State $end\_one\_sinds.add(key)$
			\EndIf
		\EndFor
		\EndProcedure
	\end{algorithmic}
\end{algorithm}

A complexidade de decodificação do sistema é  $\mathcal{O}(n^2h)$. Em que $h$ é o tamanho do conjunto de síndromes. 